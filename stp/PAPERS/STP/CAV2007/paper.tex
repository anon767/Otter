\documentclass{llncs}

\usepackage{times}

%% \usepackage{algorithm}
%% \usepackage{amssymb}
%% %\usepackage{amsthm}
%% \usepackage{graphicx}
%% \usepackage{graphics}
%% \usepackage{rotating}
%% \usepackage{caption}
%% \usepackage{wrapfig}
%% \usepackage{graphicx}
%% \usepackage{floatflt}
%% \usepackage{amsmath}
%% \usepackage{url}
%% \usepackage{subfigure}

\newcommand{\stpread}{{\it read}}
\newcommand{\stpwrite}{{\it write}}
\newcommand{\stpite}{{\it ite}}
\newcommand{\stptrue}{{\it TRUE}}
\newcommand{\stpfalse}{{\it FALSE}}

%% put in the right version of the command depending on whether you want to see comments 
%% or not.

\newcommand{\comment}[1]{{{\bf Comment:}\em #1}}
%% \newcommand{\comment}[1]{}


\date{}
\title{A Decision Procedure for Bit-Vectors and Arrays}

\author{
Vijay Ganesh and David L. Dill}
%Computer Systems Laboratory \\
%Stanford University \\
%\{vganesh, dill\} @cs.stanford.edu

\institute{Computer Systems Laboratory \\ 
           Stanford University  \\
	   \{vganesh, dill\} @cs.stanford.edu
         }

\begin{document}
%\vspace{-.1in}
\maketitle

\begin{abstract}
STP is a decision procedure for the satisfiability
of quantifier-free formulas in the theory of bit-vectors and arrays
that has been optimized for large problems encountered in software
analysis applications.  The basic architecture of the procedure
consists of word-level pre-processing algorithms followed by
translation to SAT.  The primary bottlenecks in software verification
and bug finding applications are large arrays and linear bit-vector
arithmetic. New algorithms based on the abstraction-refinement
paradigm are presented for reasoning about large arrays. A solver for
bit-vector linear arithmetic is presented that eliminates variables
and parts of variables to enable other transformations, and reduce the
size of the problem that is eventually received by the SAT solver.

These and other algorithms have been implemented in STP, which has
been heavily tested over thousands of examples obtained from several
real-world applications.  Experimental results indicate that the above
mix of algorithms along with the overall architecture is far more
effective, for a variety of applications, than a direct translation of
the original formula to SAT or other comparable decision procedures.

\end{abstract}

\section{Introduction}

Decision procedures for fragments of first-order logic are
increasingly being used in modern hardware verification and theorem
proving tools. These decision procedures usually support integer and
real arithmetic, uninterpreted functions, bit-vectors, and
arrays. Examples of such decision procedures include Yices, SVC, CVC
Lite,UCLID~\cite{yices,svc,cvcl,uclid}. Although theorem-proving and
hardware verification have been the primary users of decision
procedures, increasingly they are being used in large-scale program
analysis, bug finding and test generation
tools~\cite{exe-ccs06,replayer06}. These tools often symbolically
analyze code and generate constraints for the decision procedure to
solve, and use the results to further guide the analysis or generate
new testcases.

Software analysis tools create demands on decision procedures that are
different from those imposed by hardware applications. These
applications often generate very large array constraints, especially
when tools choose to model system memory as one or more arrays.  Also,
software analysis tools need to be able to reason about bit-vectors,
and especially mod-$2^n$ arithmetic, which is an important source of
incorrect system behavior.  The constraint problems are large and
extremely challenging to solve.

This paper reports on STP, a decision procedure for quantifier-free
first order logic with bit-vector and array
datatypes~\cite{stumparray}.  The design of STP is has been driven
primarily by the demands of software analysis research projects. STP
is being used in several software analysis, bug finding and hardware
verification applications. Notable applications include the EXE
project~\cite{exe-ccs06} at Stanford, which generates test cases for C
programs using symbolic execution, and uses STP to solve the
constraints. Other projects include the Replayer
project~\cite{replayer06} and Minesweeper~\cite{minesweeper2007} at
Carnegie Mellon University which produce constraints from symbolic
execution of machine code, and the CATCHCONV
project~\cite{catchconv07} at Berkeley which tries to catch errors due
to type conversion in C programs. The CATCHCONV project produced the
largest example solved by STP so far. It is a 412 Mbyte formula, with
2.12 million 32 bit bit-vector variables, array write terms which are
tens of thousands of levels deep, a large number of array reads with
non-constant indices (corresponding to aliased reads in memory), many
linear constraints, and liberal use of bit-vector functions and
predicates, and STP solves it in approx. 2 minutes on a 3.2GHz Linux
box.

There is a nice overview of bit-vector decison procedures
in~\cite{uclid07}, which we do not repeat here.  STP's architecture is
different from most decision procedures that support both bit-vectors
and arrays~\cite{cvc,cvcl,yices}, which are based on backtracking and
a framework for combining specialized theories such as
Nelson-Oppen~\cite{nelsonoppen}. Instead, STP consists of a series of
word-level transformations and optimizations that eventually convert
the original problem to a conjunctive-normal form (CNF) formula for
input to a high-speed solver for the satisfiability problem for
propositional logic formulas (SAT)~\cite{minisat}.  Thus, STP fully
exploits the speed of modern SAT solvers while also taking advantage
of theory-specific optimizations for bit-vectors and arrays.  In this
respect, STP is most similar to UCLID~\cite{uclid}.

The goal of this paper is to describe the factors that enable STP to
handle the large constraints from software applications.  In some
cases, simple optimizations or a careful decision about the ordering
of transformations can make a huge difference in the capacity of the
tool.  In other cases, more sophisticated optimizations are required.
Two are discussed in detail: An on-the-fly solver for mod-$2^n$ linear
arithmetic, and abstraction-refinement heuristics for array
expressions. The rest of the paper discusses the architecture of STP,
the basic engineering principles, and then goes into more detail about
the optimizations for bit-vector arithmetic and arrays. Performance on
large examples is discussed, and there is a comparative evaluation
with Yices~\cite{yices}, that is well-known for its efficiency.


\label{sec:overview}
\section{STP Overview}

STP's input language has most of the functions and predicates
implemented in a programming language such as C or a machine
instruction set, except that it has no floating point datatypes or
operations. The current set of operations supported include
$\stptrue$, $\stpfalse$, propositional variables, arbitrary Boolean
connectives, bitwise Boolean operators, extraction, concatenation,
left and right shifts, addition, multiplication, unary minus, (signed)
division and modulo, array read and write functions, and relational
operators. The semantics parallel the semantics of the SMTLIB
bit-vector language~\cite{smtlib} or the C programming language, except
that in STP bit-vectors can have any positive length. Also, all
arithmetic and bitwise Boolean operations require that the inputs be
of the same length. STP can be used as a stand-alone program, and can
parse input files in a special human readable syntax and also the
SMTLIB QF\_UFBV32 syntax~\cite{smtlib}. It can also be used as a
library, and has a special C-language API that makes it relatively
easy to integrate with other applications.

STP converts a decision problem in its logic to propositional CNF,
which is solved with a high-performance off-the-shelf CNF SAT solver,
MiniSat~\cite{minisat} (MiniSat has a nice API, and it is concise,
clean, efficient, reliable, and relatively unencumbered by licensing
conditions).  However, the process of converting to CNF includes many
word-level transformations and optimizations that reduce the
difficulty of the eventual SAT problem.  Problems are frequently
solved during the transformation stages of STP, so that SAT does not
need to be called.

STP's architecture differs significantly from many other decision
procedures based on case splitting and backtracking, including tools
like SVC, and CVC Lite~\cite{svc,cvcl}, and other solvers
based on the Davis-Putnam-Logemann-Loveland (DPLL(T))
architecture~\cite{ganzinger04dpllt}.  Conceptually, those solvers
recursively assert atomic formulas and their negations to a
theory-specific decision procedures to check for consistency with
formulas that are already asserted, backtracking if the current
combination of assertions is inconsistent.  In recent versions of this
style of decision procedure, the choice of formulas to assert is made
by a conventional DPLL SAT solver, which treats the formulas as
propositional variables until they are asserted and the decision
procedures invoked.

Architectures based on assertion and backtracking invoke
theory-specific decision-procedures in the ``inner loop'' of the SAT
solver.  However, modern SAT solvers are very fast largely because of
the incredible efficiency of their inner loops, and so it is difficult
with these architectures to take the best advantage of fast SAT
solvers.

STP on the other hand does all theory-specific processing {\em
before\/} invoking the SAT solver.  The SAT solver works on a purely
propositional formula, and its internals are not modified, including
the highly optimized inner loop.  Optimizing transformations are
employed before the SAT solver when they can solve a problem more
efficiently than the SAT solver, or when they reduce the difficulty of
the problem that is eventually presented to the SAT solver.

DPLL(T) solvers often use Nelson-Oppen combination~\cite{nelsonoppen},
or variants thereof, to link together multiple theory-specific
decision procedures. Nelson-Oppen combination needs the individual
theories to be disjoint, stably-infinite and requires the exchange of
equality relationships deduced in each individual theory, leading to
inflexibility and implementation complexity.  In return, Nelson-Oppen
ensures that the combination of theories is complete. STP is complete
because the entire formula is converted by a set of satisfiability
preserving steps to CNF, the satisfiability of which is decided by the
SAT solver.  So there is no need to worry about meeting the conditions
of Nelson-Oppen combination. Furthermore, the extra overhead of
communication between theories in the Nelson-Oppen style decision
procedures can become a bottleneck for the very large inputs that we
have seen, and this overhead is avoided in STP.

The STP approach is not always going to be superior to a good
backtracking solver.  A good input to STP is a conjunction of many
formulas that enable local algebraic transformations.  On the other
hand, formulas with top-level disjunctions may be very
difficult. Fortunately, the software applications used by STP tend to
generate large conjunctions, and hence STP's approach has worked well
in practice.

\input{epsf}
\begin{figure}
\center
\epsfysize=1.5in
\epsfxsize=3in
\epsffile{STP.eps}
\caption{STP Architecture}
\label{stparch}
\end{figure}

In more detail, STP's architecture is depicted in Figure
\ref{stparch}.  Processing consists of three phases of word-level
transformations; followed by conversion to a purely Boolean formula
and Boolean simplifications (this process is called ``Bit Blasting'');
and finally conversion to propositional CNF and solving by a SAT
solver.  The primary focus of this paper is on word level
optimizations for arithmetic, arrays and refinement for arrays.

Expressions are represented as directed acyclic graphs (DAGs), from
the time they are created by the parser or through the C-interface,
until they are converted to CNF.  In the DAG representation,
isomorphic subtrees are represented by a single node, which may be
pointed to by many parent nodes.  This representation has advantages
and disadvantages, but the overwhelming advantage is compactness.

It is possible to identify some design principles that have worked
well during the development of STP.  The overarching principle is to
procrastinate when faced with hard problems.  That principle is
applied in many ways. Transformations that are risky because they can
significantly expand the size of the expression DAG are postponed until
other, less risky, transformations are performed, in the hope that the
less risky transformation will reduce the size and number of
expressions requiring more risky transformations.  This approach is
particularly helpful for array expressions.

Counter-example-guided abstraction/refinement is now a standard
paradigm in formal tools, which can be applied in a variety of ways.
It is another application of the procrastination principle.
For example, the UCLID tool abstracts and refines the precision of
integer variables.

A major novelty of STP's implementation is the particular
implementation of the refinement loop in Figure \ref{stparch}.  In
STP, abstraction is implemented (i.e. an {\it abstract formula} is
obtained) by omitting conjunctive constraints from a {\em concrete
formula}, where the concrete formula must be equisatisfiable with the
original formula. (Logical formulas $\phi$ and $\psi$ are
equisatisfiable iff $\phi$ is satisfiable exactly when $\psi$ is
satisfiable.)
  
When testing an abstract formula for satisfiability, there can be
three results.  First, STP can determine that the abstracted formula
is unsatisfiable.  In this case, it is clear that the original formula
is unsatisfiable, and hence STP can return ``unsatisfiable'' without
additional refinement, potentially saving a massive amount of work.

A second possible outcome is that STP finds a satisfying assignment to
the abstract formula.  In this case, STP converts the satisfying
assignment to a (purported) concrete model,~\footnote{A model is an
assignment of constant values to all of the variables in a formula
such that the formula is {\it satisfied}} and also assigns zero to any
variables that appear in the original formula but not the abstract
formula, and evaluates the original formula with respect to the
purported model.  If the result of the evaluations is $\stptrue$, the
purported model is truly a model of the original formula (i.e. the
original formula is indeed satisfiable) and STP returns the model
without further refinement iterations.

The third possible outcome is that STP finds a purported model, but
evaluating the original formula with respect to that model returns
$\stpfalse$.  In that case, STP refines the abstracted formula by
heuristically choosing additional conjuncts, at least one of which
must be false in the purported model and conjoining those formulas
with the abstracted formula to create a new, less abstract formula.
In practice, the abstract formula is not modified; instead, the new
formulas are bit-blasted, converted to CNF, and added as clauses to
the CNF formula derived from the previous abstract formula, and the
resulting CNF formula solved by the SAT solver.  This process is
iterated until a correct result is found, which must occur because, in
the worst case, the abstract formula will be made fully concrete by
conjoining every formula that was omitted by abstraction.  When all
formulas are included, the result is guaranteed to be correct because
of the equisatisfiability requirement above.


\label{sec:abstractrefine}
\section{Arrays}
As was mentioned above, arrays are used heavily in software analysis
applications, and reasoning about arrays has been a major bottleneck
in many examples. STP's input language supports one-dimensional
(non-extensional) arrays~\cite{stumparray} that are indexed by
bit-vectors and contain bit-vectors.  The operations on arrays are
$\stpread(A, i)$, which returns the value at location $A[i]$ where $A$
is an array and $i$ is an index expression of the correct type, and
$\stpwrite(A, i, v)$, which returns a new array with the same value as
$A$ at all indices except possibly $i$, where it has the value $v$.
The value of a $\stpread$ is a bit-vector, which can appear as an
operand to any operation or predicate that operates on bit-vectors.
The value of an array variable or an array write has an array type,
and may only appear as the first operand of a $\stpread$ or
$\stpwrite$, or as the then or else operand of an if-then-else.  In
particular, values of an array type cannot appear in an equality or
any other predicate.

In the unoptimized mode, STP reduces all formulas to an
equisatisfiable form that contains no array $\stpread$s or
$\stpwrite$s, using three transformations.  (In the following, the
expression $\stpite(c_1,e_1,e_2)$ is shorthand for {\it if $c_1$ then
$e_1$ else $e_2$ endif}.)  These transformations are all standard.

The {\bf Ite-lifting\/} transformation converts
$\stpread(\stpite(c, \stpwrite(A, i, v), e), j)$ to
$\stpite(c,$ $\stpread(\stpwrite(A, i, v), j), e)$.  
(There is
a similar transformation when the $\stpwrite$ is in the ``else'' part
of the $\stpite$.)  The {\bf read-over-write} transformation
eliminates all write terms by transforming 
$\stpread(\stpwrite(A, i, v), j)$ to $\stpite(i=j, v, \stpread(A, j))$.  
Finally, the {\bf read
elimination} transformation eliminates $\stpread$ terms by introducing
a fresh bit-vector variable for each such expression, and adding
more predicates to ensure consistency.  Specifically, whenever a
term $\stpread(A, i)$ appears, it is replaced by a fresh variable $v$,
and new predicates are conjoined to the formula $i = j \Rightarrow v =
w$ for all variables $w$ introduced in place of read terms
$\stpread(A, j)$, having the same array term as first operand. As an
example of this transformation, the simple formula $(\stpread(A, 0) =
0) \wedge (\stpread(A, i) = 1)$ would be transformed to
$v_1 = 0 \wedge v_2 = 1 \wedge (i = 0 \Rightarrow v_1 = v_2)$.
The formula of the form $(i = 0 \Rightarrow v_1 = v_2)$ is 
called an {\it array read axiom}.

\subsection{Optimizing array reads}

Read elimination, as described above, expands each formula by up to
$n(n-1)/2$ nodes, where $n$ is the number of syntactically distinct
index expressions. Unfortunately, software analysis applications can
produce thousands of reads with variable indices, resulting in a
lethal blow-up when this transformation is applied.  
While this blow-up seems unavoidable in the worst case,
appropriate procrastination leads to practical solutions for many very
large problems. Two optimizations which have been very
effective are {\it array substitution} and abstraction-refinement for
reads, which we call {\it read refinement}.

The array substitution optimization reduces the number of array
variables by substituting out all constraints of the form $\stpread(A,
c) = e_1$, where $c$ is a constant and $e_1$ does not contain another
array read.  Programs often index into arrays or memory using constant
indexes, so this is a case that occurs often in practice.

The optimization has two passes. The first pass builds a substitution
table with the left-hand-side of each such equation ($\stpread(A, c)$)
as the key and the right-hand-side ($e_1$) as the value, and then
deletes the equation from the input query. The second pass over the
expression replaces each occurrence of a key by the corresponding
table entry. Note that for soundness, if a second
equation is encountered whose left-hand-side is already in the table, 
the second equation is not deleted and the table is not changed.
For example, if STP saw $\stpread(A, c) = e_1$ then $\stpread(A, C) = e_2$,
the second formula would not be deleted and would later be simplified
to $e_1 = e_2$.

The second optimization, {\it read refinement}, delays the translation
of array {\stpread}s with non-constant indexes in the hope of avoiding
read elimination blowup. Its main trick is to solve a less-expensive
approximation of the formula, check the result in the original
formula, and try again with a more accurate approximation if the
result is incorrect.

Read formulas are abstracted by performing read elimination,
{\em i.e.,} replacing reads with new variables, but not adding the array read
axioms.  This abstracted formula is processed by the remaining
stages of STP.  As discussed in the overview, if the result is
unsatisfiable, that result is correct and can be returned immediately
from STP.  If not, the abstract model found by STP is converted to a
concrete model and the original formula is evaluated with respect to
that model.  If the result is $\stptrue$, the answer is correct and
STP returns that model.  Otherwise, some of the array read axioms from
read elimination are added to the formula and STP is asked to satisfy
the modified formula.
This iteration repeats until a correct result is found,
which is guaranteed to happen (if memory and time are not exhausted)
because all of the finitely many array read axioms will eventually be
added in the worst case.

The choice of which array read axioms to add during refinement is a
heuristic that is important to the success of the method.  A policy
that seems to work well is to find a non-constant array index term for
which at least one axiom is violated, then add all of the violated
axioms involving that term.  Adding at least one false axiom during
refinement guarantees that STP will not find the same false model more
than once.  Adding all the axioms for a particular term seems
empirically to be a good compromise between adding just one formula,
which results in too many iterations, and adding all formulas, which
eliminates all abstraction after the first failure.

For example, suppose STP is given the formula $(\stpread(A, 0) = 0)
\wedge (\stpread(A, i) = 1)$.  STP would first apply the substitution
optimization by deleting $\stpread(A, 0) = 0$ from the formula, and
inserting the pair $(\stpread(A, 0), 0)$ in the substitution
table. Then, it would replace $\stpread(A, i)$ by a new variable
$v_i$, thus generating the under-constrained formula $v_i = 1$.
Suppose STP finds the solution $i = 1$ and $v_i = 1$.

STP then translates the solution to the variables of the original
formula to get $(\stpread(A, 0) = 0)$ $\wedge$ $\stpread(A, 1) = 1)$.
This solution is satisfiable in the original formula as well, so STP
terminates since it has found a true satisfying assignment.

However, suppose that STP finds the solution $i = 0$ and $v_i = 1$.
Under this solution, the original formula eventually evaluates to
$\stpread(A,0) = 0 \wedge \stpread(A, 0) = 1$, which after
substitution gives $0=1$. Hence, the solution to the under-constrained
formula is not a solution to the original formula.

In this case, STP adds the array read axiom $i=0 \Rightarrow
\stpread(A, i) = \stpread(A, 0)$.  When this formula is checked, the
result must be correct because the new formula includes the complete
set of array read axioms.

\subsection{Optimizing array writes}

Efficiently dealing with array writes is crucial to STP's utility in
software applications, some of which produce deeply nested write terms
when there are many successive assignments to indices of the same array.
The {\bf read-over-write} transformation
creates a performance bottleneck by destroying sharing of subterms,
creating an unacceptable blow-up in DAG size.
Consider the simple formula:
$\stpread(\stpwrite(A,i,v),j) = \stpread(\stpwrite(A,i,v),k)$, in
which the $\stpwrite$ term is shared.

The {\bf read-over-write} transformation translates this to
$\stpite(i=j,v,\stpread(A,j)) = \stpite(i=k,v,\stpread(A,k))$.  When
applied recursively to the deeply nested $\stpwrite$ terms, it
essentially creates a new copy of the entire DAG of write terms for
every distinct read index, which exhausts memory in large examples.

Once again, the procrastination principle applies.  
The {\bf read-over-write} transformation is delayed until after other
simplification and solving transformations are performed,
except in special cases like $\stpread(\stpwrite(A, i, v), i+1)$,
where the read and write indices simplify to terms that are 
always equal or not equal.
In practice, the simple transformations convert many index terms to
constants.  The {\bf read-over-write} transformation is applied
in a subsequent phase.  When that happens, the formula is smaller
and contains more constants.
This simple optimization is enormously effective, enabling STP to
solve many very large problems with nested writes that it is otherwise
unable to do.

Abstraction and refinement can also be used on write expressions, when
the previous optimization leaves large numbers of $\stpread$s and
$\stpwrite$s, leading to major speed-ups on some large formulas.
For this optimization, array read-over-write terms are
replaced by new variables to yield a conjunction of formulas that is
equisatisfiable to the original set.  The example above is transformed
to:
\[
\begin{array}{rcl}
v_1 &=& v_2 \\
v_1 &=& \stpite(i=j,v,\stpread(A,j)) \\
v_2 &=& \stpite(i=k,v,\stpread(A,k))
\end{array}
\]
where the last two formulas are called {\it array write axioms}. For
the abstraction, the array write axioms are omitted, and the
abstracted formula $v_1 = v_2$ is processed by the remaining phases of
STP.  As with array reads, the refinement loop iterates only if STP
finds a model of the abstracted formula that is also not a model of
the original formula. Write axioms are added to the abstracted
formula, and the refinement loop iterates with the additional axioms
until a definite result is produced. Although, this technique leads to
improvement in certain cases, the primary problem with it is that the
number of iterations of the refinement loop is sometimes very large.


\label{sec:linearsolver}
\section{Linear Solver and Variable Elimination}

One of the essential features of STP for software analysis
applications is its efficient handling of linear twos-complement
arithmetic.  The heart of this is an {\em on-the-fly\/} solver.  The
main goal of the solver is to eliminate as many bits of as many
variables as possible, to reduce the size of the transformed problem
for the SAT solver.  In addition, it enables many other
simplifications, and can solve purely linear problems outright, so
that the SAT solver does not need to be used.

The solver solves for one equation for one variable at a time.  That
variable can then be eliminated by substitution in the rest of the
formula, whether the variable occurs in linear equations or other
formulas.  In some cases, it cannot solve an entire variable, so it
solves for some of the low-order bits of the variable.  After
bit-blasting, these bits will not appear as variables in the problem
presented to the SAT solver. Non-linear or word-level terms (extracts,
concats etc.) appearing in linear equations are treated as bit-vector
variables.

The algorithm has worst-case time running time of $O(k^2n)$
multiplications, where $k$ is the number of equations and $n$ is the
number of variables in the input system of linear bit-vector
equations.\footnote{As observed in ~\cite{BDL98}, the theory of linear
mod $2^n$ arithmetic (equations only) in tandem with concatenate and
extract operations is NP-complete. Although STP has concatenate and
extraction operations, terms with those operations are treated as
independent variables in the linear solving process, which is
polynomial. 

A hard NP-complete input problem constructed out of linear operations,
concatenate and extract operations will not be solved completely by
linear solving, and will result in work for the SAT solver.}  If the
input is unsatisfiable the solver terminates with $\stpfalse$. If the
input is satisfiable it terminates with a set of equations in
\textit{solved form}, which symbolically represent all possible
satisfying assignments to the input equations.  So, in the special
case where the formula is a system of linear equations, the solver
leads to a sound and complete polynomial-time decision procedure.
Furthermore, the equations are reduced to a closed form that captures
all of the possible solutions.

\begin{definition}
\emph{Solved Form:} A list of equations is in
\textit{solved form} if the following invariants hold over the
equations in the list.

1) Each equation in the list is of the form $x[i:0] = t$ or $x=t$,
where $x$ is a variable and $t$ is a linear combination of the
variables or constant times a variable (or extractions thereof)
occuring in the equations of the list, except $x$

2) Variables on the left hand side of the equations occuring earlier
in the list may not occur on the right hand side of subsequent
equations. Also, there may not be two equations with the same left
hand side in the list

3) If extractions of variables occur in the list, then they must
always be of the form $x[i:0]$, i.e. the lower extraction index must
be 0, and all extractions must be of the same length

4) If an extraction of a variable $x[i:0]= t$ occurs in the list, then
   an entry is made in the list for $x=x^{1}@t$, where $x^1$ is a new
   variable refering to the top bits of $x$ and $@$ is the
   concatenation symbol
\end{definition}

The algorithm is illustrated on the following system:
\begin{eqnarray}
3x + 4y + 2z &=& 0\nonumber\\
2x + 2y + 2 &=& 0 \nonumber\\
4y + 2x + 2z &=& 0\nonumber
\end{eqnarray}
where all constants, variables and functions are 3 bits long. 

The solver proceeds by first choosing an equation and always checks if
the chosen equation is {\it solvable}. It uses the following theorem
from basic number theory to determine if an equation is solvable:
$\Sigma_{i=1}^n a_ix_i = c_i$ mod $2^b$ is solvable for the unknowns
$x_i$ if and only if the greatest common divisor of
\{$a_1,\ldots,a_n,2^b$\} divides $c_i$.

In the example above, the solver chooses $3x + 4y + 2z = 0$ which is
solvable since the $gcd(3,4,2,2^3)$ does indeed divide $0$. It is also
a basic result from number theory that a number $a$ has a
multiplicative inverse mod $m$ iff $\gcd(a, m) = 1$, and that this
inverse can be computed by the extended greatest-common divisor
algorithm~\cite{CLR} or a method from~\cite{BDL98}. So, if there is a
variable with an odd coefficient, the solver isolates it on the
left-hand-side and multiplies through by the inverse of the
coefficient.  In the example, the multiplicative inverse of $3$ mod
$8$ is also $3$, so $3x + 4y + 2z = 0$ can be solved to yield $x = 4y
+ 6z$.

Substituting $4y+6z$ for $x$ in the remaining two equations yields the
system 
\begin{eqnarray}
2y + 4z + 2 &=& 0 \nonumber\\
4y + 6z &=& 0\nonumber
\end{eqnarray}

where all coefficients are even. Note that even coefficients do not
have multiplicative inverses in arithmetic mod $2^b$, and, hence we
cannot isolate a variable. However, it is possible to solve for {\em
some bits\/} of the remaining variables.

The solver transforms the whole system of solvable equations into a
system which has at least one summand with an odd coefficient. To do
this, the solver chooses an equation which has a summand whose
coefficient has the minimum number of factors of 2. In the example,
this would the equation $2y + 4z + 2 =0$, and the summand would be
$2y$. The whole system is divided by 2, and the high-order bit of each
variable is dropped, to obtain a reduced set of equations

\begin{eqnarray}
 y[1:0] + 2z[1:0] + 1 &=& 0 \nonumber\\
2y[1:0] + 3z[1:0] &=& 0 \nonumber\
\end{eqnarray}

where all constants, variables and operations are 2 bits.  Next,
$y[1:0]$ is solved for to obtain $y[1:0] = 2z[1:0] + 3$. Substituting
for $y[1:0]$ in the system yields a new system of equations $3z[1:0] +
2 = 0$. This equation can be solved for $z[1:0]$ to obtain $z[1:0] =
2$. It follows that original system of equations is satisfiable. It is
important to note here that the bits $y[2:1]$ and $z[2:1]$ are
unconstrained. The solved form in this case is $x=4y+6z$ $\wedge$
$y[1:0] = 2z[1:0] + 3$ $\wedge$ $z[1:0] = 2$ (Note that in the last
two equations all variables, constants and functions are 2 bits long).

Algorithms for deciding the satisfiability of a system of equations
and congruences in modular or residue arithmetic have been well-known
for a long time. However, most of these algorithms do not provide a
solved form that captures all possible solutions.  Some of the ideas
presented here were devised by Clark Barrett and implemented in the
SVC decision procedure~\cite{cheng01,BDL98}, but the SVC algorithm has
exponential worst-case time complexity while STP's linear solver is
polynomial in the worst-case.

The closest related work is probably in a paper by Huang and
Cheng~\cite{cheng01}, which reduces a set of equations to a solved
form by Guassian elimination. On the other hand, STP implements an
online solving and substitution algorithm that gives a closed form
solution. Such algorithms are easier to integrate into complex
decision procedures.


\label{sec:results}
\section{Experimental Results}

STP has been thoroughly tested over thousands of examples derived from
around a dozen different projects in software analysis, bug finding
and hardware verification. Some notable examples include: 1) {\bf EXE}
: Bug finding tool from Stanford~\cite{exe-ccs06} 2) {\bf CATCHCONV} :
A tool for catching conversion errors from Berkeley~\cite{catchconv07}
3) {\bf Replayer} : Automatic binary analysis of protocols from
CMU~\cite{replayer06} 4) {\bf Minesweeper} : Another bug finding tool
from CMU~\cite{minesweeper2007}.

Unfortunately, we could not translate most of our large (called {\it
BigArray Examples} in the logs) benchmarks into SMTLIB format because
the CVC2SMT translator available at~\cite{smtlib} segfaults on these
examples. This prevents us from comparing STP with Yices on the
BigArray examples. Therefore, we present three tables. In Table
\ref{STP-table} STP is compared with all optimizations on (All ON),
Array Optimizations on (Arr-ON,Lin-OFF), linear-solving on
(Arr-OFF,Lin-ON), and all optimizations off (ALL OFF) on the BigArray
examples (these examples are heavy on linear arithmetic and array
reads). In Table \ref{STP-write-abs} we present STP's performance on
big array examples with deeply nested writes. In Table
\ref{STP-Yices-table} we present the comparison of STP with Yices on a
very small version of a BigArray example, and some randomly generated
linear system of equations. Even though the linear solving examples
are randomly generated, they give a clear indication of how Yices will
perform on the real-world inputs since they require good solving. All
experiments were run on a 3.2GHz/2GB RAM Intel machine running Linux.

\begin{table}[t]
%\footnotesize
\begin{center}
\begin{tabular}{|l|l|l|l|l|l|}
\hline
{\bf Example Name (Node Size)} & {\bf Result} & {\bf All ON} & {\bf Arr-ON,Lin-OFF} & {\bf Arr-OFF,Lin-ON} & {\bf All OFF}\\
\hline
testcase15 (0.9M) & sat & 66 & 192 & 64  & MO \\
testcase16 (0.9M) & sat & 67 & 233 & 66  & MO \\
thumbnailout-spin1 (3.2M)& sat & 115 & 111 & 113 & MO \\
thumbnailout-spin1-2 (4.3M)& NR & MO & MO & MO  & MO \\
thumbnailout-noarg (2.7M)& sat & 840 & MO & MO  & MO \\
\hline
\end{tabular}
\end{center}
\caption{STP performance in different modes over BigArray
Examples. Names are followed by the nodesize. Approximate nodesize is
in millions of nodes. 1M is one million nodes. Shared nodes are
counted exactly once. NR stands for No Result.  All timings are in
seconds. MO stands for out of memory error. These examples were
generated using the CATCHCONV tool}
\label{STP-table}
\end{table}

For Table \ref{STP-table} we chose some of the hardest of the BigArray
examples which are usually 10s of MBytes of text, typically 100s of
thousands of 32 bit bit-vector variables, lots of array reads, and
large number of linear constraints derived
from~\cite{catchconv07,replayer06}. The primary reason for timeout was
usually because of out-of-memory exception. Table \ref{STP-table}
shows that all optimizations are required for solving the hardest
realword problems. As expected STP's linear solver is the most
effective in solving these examples.

\begin{table}[t]
%\footnotesize
\begin{center}
\begin{tabular}{|l|l|l|l|l|l|}
\hline
{\bf Example Name (Node Size)} & {\bf Result} & {\bf WRITE Abstraction} & {\bf NO WRITE Abstraction} \\
\hline
grep0084   (69K) & sat & 109 & 506 \\
grep0095   (69K) & sat & 115 & 84 \\
grep0106   (69K) & sat & 270 & $>$ 600 \\
grep0117   (70K) & sat & 218 & $>$ 600 \\
grep0777   (73K) & NR  &  MO & MO \\
610dd9dc   (15K) & sat & 188 & 101 \\
testcase20 (1.2M)& sat & 67  & MO \\
\hline
\end{tabular}
\end{center}
\caption{STP performance in different modes over BigArray Examples
with deep nested writes. Names are followed by the nodesize. 1M is one
million nodes (1K is thousand nodes). Shared nodes are counted exactly
once. NR stands for No Result.  All timings are in seconds. MO stands
for out of memory error.These examples were generated using the
CATCHCONV and MINESWEEPER tools}
\label{STP-write-abs}
\end{table}

For Table \ref{STP-write-abs} we chose examples with lots of deeply
nested array writes and modest amounts of linear constraints derived
very different applications. The grep examples were generated using
the MINESWEEPER tool while trying to find bugs in unix grep program.
The 610dd9c formula is generated by a MINESWEEPER analysis of a
program that is used in botnet attack. The formula testcase20 was
generated by CATCHCONV which does analysis on the VALGRIND IR of C
programs. As expected STP with write abstraction-refinement ON gives a
very large improvement over STP with write abstraction-refinement
switched OFF.

\begin{table}[t]
\footnotesize
\begin{center}
\begin{tabular}{|l|r|r|}
\hline
{\bf Example} & {\bf STP} & {\bf Yices} \\
\hline
25 var/25 equations(unsat) &  0.8s  & 42s  \\
50 var/50 equations(sat)   &  13s  &  TimeOut \\
cookie checksum example(sat) &  2.6s &  218s \\
\hline
\end{tabular}
\end{center}
\caption{STP vs. Yices. Timeout per example: 600sec. The last example
was generated using the Replayer tool}
\label{STP-Yices-table}
\end{table}

We compare Yices~\cite{yices}, one of the best SMTLIB solvers, with
STP on one of the smallest real-world BigArray examples that we
managed to translate using CVC2SMT translator (As mentioned before all
these examples have very similar characterisitics even though they are
generated from very different applications).  Yices is nearly 100x
slower than STP. In order to compare the linear solving of Yices and
STP, we generated some random examples with very small number of
variables and equations, where each variable is a 32 bit
bit-vector. We present two cases for illustration in Table
\ref{STP-Yices-table}. Yices timesout on even a 50 variable 50
equation example, and when it does finish it is much slower than
STP. The STP website~\cite{stpwebsite} provides more complete
regression logs, where STP and Yices are compared on all examples that
we could translate to SMTLIB format.



\label{sec:conclusion}
\section{Conclusion}
Software applications such as program analysis, bug finding, and
symbolic simulation of software tend to impose different conditions on
decision procedures than hardware applications. In particular, arrays
become a bottleneck. Also, the constraints tend to be very large with
lots of linear bit-vector arithmetic in them. 
Abstraction-refinement algorithms 
is often helpful for handling large array terms.
Also, the approach of doing phased word-level
transformations, starting with the least expensive and risky 
transformations, followed by translation
to SAT seems like a good design for decision procedures for the
applications considered. Finally, linear solving, when implemented
carefully, is effective in variable elimination.



\section*{Acknowledgements}
We are indebted to the following users for their feedback and for
great examples: David Molnar from Berkeley; Cristian Cadar, Dawson
Engler and Aaron Bradley from Stanford; Jim Newsome, David Brumley,
Ivan Jaeger and Dawn Song from CMU;

This research was supported by Department of Homeland Security (DHS)
grant FA8750-05-2-0142.

{\bibliographystyle{abbrv} \bibliography{biblio} }
\end{document}
