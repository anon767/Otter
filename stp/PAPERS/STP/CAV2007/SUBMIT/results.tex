\section{Experimental Results}

This section presents empirical results on large examples from
software analysis tools, and on randomly generated sets of linear
equations.  The effects of abstraction and linear solving in STP are
examined.  It is difficult to compare STP with other decision
procedures, because no publicly available decision procedures except
CVCL (from the authors research group) can deal with terms involving
both bit-vectors and arrays indexed by bit-vectors.  CVCL is
hopelessly inefficient compared with STP, which was written to replace
it.  Terms in Yices can include bit-vectors and uninterpreted
functions over bit-vectors.  Uninterpreted functions are equivalent to
arrays with no $\stpwrite$ operations, so it is possible to compare
the performance of STP and Yices on examples with linear arithmetic
and one realistic example with a read-only array.

In Table
\ref{STP-table}, STP is compared with all optimizations on (All ON),
Array Optimizations on (Arr-ON,Lin-OFF), linear-solving on
(Arr-OFF,Lin-ON), and all optimizations off (ALL OFF) on the BigArray
examples (these examples are heavy on linear arithmetic and array
reads). Table \ref{STP-write-abs} summarizes STP's performance, with and
without array write abstraction, on
the big array examples with deeply nested writes. Table
\ref{STP-Yices-table} compares  STP with Yices on a
very small version of a BigArray example, and some randomly generated
linear system of equations. 
All experiments were run on a 3.2GHz/2GB RAM Intel machine running Linux.

\begin{table}[t]
\begin{center}
\begin{tabular}{|l|l|l|l|l|l|}
\hline
{\bf Example Name (Node Size)} & {\bf Result} & {\bf All ON} & {\bf Arr-ON,Lin-OFF} & {\bf Arr-OFF,Lin-ON} & {\bf All OFF}\\
\hline
testcase15 (0.9M) & sat & 66 & 192 & 64  & MO \\
testcase16 (0.9M) & sat & 67 & 233 & 66  & MO \\
thumbnailout-spin1 (3.2M)& sat & 115 & 111 & 113 & MO \\
thumbnailout-spin1-2 (4.3M)& NR & MO & MO & MO  & MO \\
thumbnailout-noarg (2.7M)& sat & 840 & MO & 840  & MO \\
\hline
\end{tabular}
\end{center}
\caption{STP performance in different modes over BigArray
Examples. Names are followed by the nodesize. Approximate node size is
in millions of nodes. 1M is one million nodes. Shared nodes are
counted exactly once. NR stands for No Result.  All timings are in
seconds. MO stands for out of memory error. These examples were
generated using the CATCHCONV tool}
\label{STP-table}
\end{table}

Table \ref{STP-table} includes some of the hardest of the BigArray
examples which are usually tens of megabytes of text, typically hundreds of
thousands of 32 bit bit-vector variables, lots of array reads, and
large number of linear constraints derived
from~\cite{catchconv07,replayer06}. The primary reason for timeouts is
an out-of-memory exception. Table \ref{STP-table}
shows that all optimizations are required for solving the hardest
real-world problems. As expected, STP's linear solver is very helpful
in solving these examples.

\begin{table}[t]
\begin{center}
\begin{tabular}{|l|l|l|l|l|l|}
\hline
{\bf Example Name (Node Size)} & {\bf Result} & {\bf WRITE Abstraction} & {\bf NO WRITE Abstraction} \\
\hline
grep0084   (69K) & sat & 109 & 506 \\
grep0095   (69K) & sat & 115 & 84 \\
grep0106   (69K) & sat & 270 & $>$ 600 \\
grep0117   (70K) & sat & 218 & $>$ 600 \\
grep0777   (73K) & NR  &  MO & MO \\
610dd9dc   (15K) & sat & 188 & 101 \\
testcase20 (1.2M)& sat & 67  & MO \\
\hline
\end{tabular}
\end{center}
\caption{STP performance in different modes over BigArray Examples
with deep nested writes. Names are followed by the nodesize. 1M is one
million nodes (1K is thousand nodes). Shared nodes are counted exactly
once. NR stands for No Result.  All timings are in seconds. MO stands
for out of memory error.These examples were generated using the
CATCHCONV and Minesweeper tools}
\label{STP-write-abs}
\end{table}

Table \ref{STP-write-abs} includes examples with deeply
nested array writes and modest amounts of linear constraints derived
from various applications. The ``grep'' examples were generated using
the Minesweeper tool while trying to find bugs in unix grep program.
The 610dd9c formula is generated by a Minesweeper analysis of a
program that is used in ``botnet'' attack. The formula testcase20 was
generated by CATCHCONV.
As expected, STP with write abstraction-refinement ON can yield a
very large improvement over STP with write abstraction-refinement
switched OFF, although it is not always faster.

\begin{table}[t]
\footnotesize
\begin{center}
\begin{tabular}{|l|r|r|}
\hline
{\bf Example} & {\bf STP} & {\bf Yices} \\
\hline
25 var/25 equations(unsat) &  0.8s  & 42s  \\
50 var/50 equations(sat)   &  13s  &  TimeOut \\
cookie checksum example(sat) &  2.6s &  218s \\
\hline
\end{tabular}
\end{center}
\caption{STP vs. Yices. Timeout per example: 600sec. The last example
was generated using the Replayer tool}
\label{STP-Yices-table}
\end{table}

Yices and STP were also compared on small, randomly-generated systems
of linear equations with coefficients ranging from 1 to $2^{16}$, from
4 to 256 variables of 32 bits each, and 4 to 256 equations.  Yices
consistently timed out at 200 seconds on examples with 32 or more
variables, and was significantly slower than STP on the smaller
examples. The hardest problem for STP in this set of benchmarks was a
test case with 32 equations and 256 variables of 32 bits, which STP
solved in 90 seconds. There are two cases for illustration in Table
\ref{STP-Yices-table}. Yices times out on even a 50 variable 50
equation example, and when it does finish it is much slower than STP.

There is one large, real example with read-only arrays, linear
arithmetic and bit-vectors which is suitable for comparison with
Yices.  On this example, Yices is nearly one hundred times slower than
STP. Unfortunately, we could not compare Yices with STP on examples
with array writes since Yices does not support array writes with
bit-vector indexing.  More meaningful comparisons will have to wait
till competing decision procedures includes bit-vector operations and
a theory of arrays indexed by bit-vectors. All tests in this section
are available at \\ \verb|http://verify.stanford.edu/stp.html|.

